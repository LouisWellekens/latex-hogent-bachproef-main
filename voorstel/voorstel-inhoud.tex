%---------- Inleiding ---------------------------------------------------------

\section{Introductie}%
\label{sec:introductie}

\noindent Cryptocurrencies have seen a significant increase in interest over the last few years. By now almost everyone is familiar with this type of investement and a lot of people have invested themselves. This thesis will delve deeper into the possible relationship between social media and cryptocurrencies. Using data mining and analysis tools, the goal is to investigate if social media mentions and sentiment is correlated with the price of certain cryptocurrencies.  

\noindent This thesis aims to support individual and institutional crypto investors during their investment analysis. In the event that a relationship can be found between price and certain social media data, this could be of great aid to investors in order to make better decisions. 

\noindent Cryptocurrencies are new to the investing landscape which means they have some unique differences when compared to more traditional investment options. One of the big differences is the absence of financial advisors who can steer new investors into the right direction. As of yet there are very little investment institutions which offer cryptocurrency advice and guidance to their clients. In addition to this, a lot of the more established investment news sources have a troubled relationship with cryptocurrencies and don't cover them. This leads a lot of people to social media, and the internet as a whole, in order to get their investment advice. This opens up the opportunity to analyse the data which potential investors are consuming and looking for a relationship between this data and cryptocurrency prices. In this thesis we will be analysing this potential relationship and what we can conclude from it. The main research questions being :

\begin{itemize}
    \item Is there a relationship between social media data and cryptocurrency prices ?
    \item Can social media data be used in order to predict the prices of cryptocurrencies ?
    \item Is social media sentiment analysis a valuable addition to the investment thesis of investors.?
\end{itemize}



\noindent Once our research is over and our main questions are answered, the goal is to have a clear report which can be issued to cryptocurrency investors. This report should offer a clear conclusion and possibly aid them in future investing. 


%---------- Stand van zaken ---------------------------------------------------

\section{State-of-the-art}%
\label{sec:state-of-the-art}

\noindent The total number of cryptocurrency investors has soared during the last few years. Between 2017 and 2022 the number of people interacting with cryptocurrencies in some way has increased from 35 to 295 million \autocite{Statista2022}. During the peak of the cryptocurrency hype the total market cap of all coins reached over 3 trillion and the price of a single bitcoin was more than 69,000 dollars \autocite{CoinGecko2022}.  But how do people actually evaluate and decide which currency they want to invest in? Existing surveys suggest that a lot of people use social media in order to form their investment thesis \autocite{Magnusson2022}. A new term has even been coined which is the ``Finfluencer''. A finfluencer is a Financial influencer, someone who influences the financial decisions of their followers \autocite{CambridgeWords2021}. Experienced investors may be shocked by the fact that people use social media in order to create their investment thesis but this shouldn't come as a surprise. The average age of a cryptocurrency investor today is 38, about 10 years younger than the average stock market investor. \autocite{Iacurci2021}. Younger investors are more familiar with social media and have used it for most parts of their life. Reports suggest that Gen Z is 5 times more likely to get investment advice from social media and 91 \% of 18-40 year-olds report getting investment-related advice from social media \autocite{Purnell2022}. In addition to this, there are currently very few alternatives to social media for cryptocurrency investment advice. Most of the established investment institutions don't offer cryptocurrency advice and some major financial news outlets avoid the topic aswell.


\noindent This leaves us with social media like Twitter, Reddit and Youtube becoming the most popular websites for cryptocurrency discussion and investment advice. There is already evidence of social media influencing the price of a cryptocurreny. With studies showing that Elon Musks tweets can have an impact on Bitcoin's short term price action. \autocite{Huynh2022}.

\noindent Prediciting the prices of cryptocurrencies is notoriously difficult and depends on multiple factors, one of them being market sentiment\autocite{Lewis2018}. In more traditional financial markets there have also been indications that analysing sentiment can help predict future prices\autocite{Cristescu2022}. Social media platforms like Twitter should be able to provide us with an insight into market sentiment.
% Voor literatuurverwijzingen zijn er twee belangrijke commando's:
% \autocite{KEY} => (Auteur, jaartal) Gebruik dit als de naam van de auteur
%   geen onderdeel is van de zin.
% \textcite{KEY} => Auteur (jaartal)  Gebruik dit als de auteursnaam wel een
%   functie heeft in de zin (bv. ``Uit onderzoek door Doll & Hill (1954) bleek
%   ...'')



%---------- Methodologie ------------------------------------------------------
\section{Methodologie}%
\label{sec:methodologie}


\noindent In order to arrive at a suitable conclusion, Twitter  data will be collected over a 3 month period. Once the data has been gathered, we can then analyse the sentiment of that data and subsequently compare it to the evolution in price over the same period. We can run a regression analysis in order to determine a possible correlation between the variables. The dependent variable will always be the evolution in price of the cryptocurrency. There are multiple options for the independent variable, such as:

\begin{itemize}
    \item Number of postive or negative tweets concerning a cryptocurrency increase in \%
    
    \item Number of positive or negative tweets concerning a cryptocurrency in comparison to last month in \%
    
    \item Amount of likes and retweets tweets regarding the cryptocurrency receive 
    
\end{itemize}
\noindent For this research we will be using Twitter in order to gather the required data. Twitter is currently the most used platform to discuss cryptocurrencies and holds the most influential voices within the cryptocurrency market. We will be anaylysing historical tweets over a period of 3 months. These months will be September to November of 2021, during this time period the daily volume of the cryptocurrency market was at its highest ever\autocite{CoinGecko2022}.

We will be analysing three different cryptocurrencies, namely Bitcoin, Dogecoin and Solana. Bitcoin is an obvious choice as it is the most well known and holds the most value. Dogecoin is a coin which is infamous on the internet and holds supporters such as Elon Musk. Finally Solana was chosen as it was one of the cryptocurrencies which was the most volatile during the last few years, and had one of the largest \% price increases of all the cryptocurrencies.

\noindent Python will be used to collect data from twitter using the Tweepy library to access the Twitter API. In order to store and analyse the data, we will further use the Pandas, Tweepy and Matplotlib libraries. For the sentiment analysis of the data, we will mainly be using the TextBlob and VADER libraries. Finally to compare the sentiment and price actions we will be making use of logistic regression and Granger causality test.


%---------- Verwachte resultaten ----------------------------------------------
\section{Verwacht resultaat, conclusie}%
\label{sec:verwachte_resultaten}

\noindent The main result will be the conclusions we can draw from our sentiment analysis. As we will be running a regression analysis on our data we should have a final result which indicates if there is a possible relationship and which data is dependant and to what degree. If we can conclude a relationship, then our thesis could be of great benefit to potential cryptocurrency investors. The ability to gather data from social media channels is open to everyone, which means people would be able to reproduce our steps and gain an edge and improve their investing decisions. 


