%==============================================================================
% Sjabloon onderzoeksvoorstel bachproef
%==============================================================================
% Gebaseerd op document class `hogent-article'
% zie <https://github.com/HoGentTIN/latex-hogent-article>

% Voor een voorstel in het Engels: voeg de documentclass-optie [english] toe.
% Let op: kan enkel na toestemming van de bachelorproefcoördinator!
\documentclass[english]{hogent-article}

% Invoegen bibliografiebestand
\addbibresource{voorstel.bib}

% Informatie over de opleiding, het vak en soort opdracht
\studyprogramme{Professionele bachelor toegepaste informatica}
\course{Bachelorproef}
\assignmenttype{Onderzoeksvoorstel}
% Voor een voorstel in het Engels, haal de volgende 3 regels uit commentaar

\studyprogramme{Bachelor of applied information technology}
 \course{Bachelor thesis}
 \assignmenttype{Research proposal}

\academicyear{2022-2023} % TODO: pas het academiejaar aan

% TODO: Werktitel
\title{Analysing the relationship social media sentiment and cryptocurrency value }

% TODO: Studentnaam en emailadres invullen
\author{Louis Wellekens}
\email{louis.wellekens.y2144@student.hogent.be}

% TODO: Medestudent
% Gaat het om een bachelorproef in samenwerking met een student in een andere
% opleiding? Geef dan de naam en emailadres hier
% \author{Yasmine Alaoui (naam opleiding)}
% \email{yasmine.alaoui@student.hogent.be}

% TODO: Geef de co-promotor op
\supervisor[Co-promotor]{}

% Binnen welke specialisatierichting uit 3TI situeert dit onderzoek zich?
% Kies uit deze lijst:
%
% - Mobile \& Enterprise development
% - AI \& Data Engineering
% - Functional \& Business Analysis
% - System \& Network Administrator
% - Mainframe Expert
% - Als het onderzoek niet past binnen een van deze domeinen specifieer je deze
%   zelf
%
\specialisation{AI \& Data Engineering}
\keywords{Cryptocurrencies, Regression, Sentiment Analysis, Twitter, Social Media}

\begin{document}

\begin{abstract}
  The goal of this thesis is to analyse the possible relationship between social media sentiment and cryptocurrency price movements. Social media remains one of the top news sources for cryptocurrency investors which could possibly indicate a relationship between social media platforms and the investment strategies of the average investor. The goal is to simplify the sentiment of these platforms into data which can then be analysed and used to analyse market sentiment. A possible relationship would be of great benefit to cryptocurrency investors as it would greatly aid them in forming their investment thesis.
\end{abstract}

\tableofcontents

% De hoofdtekst van het voorstel zit in een apart bestand, zodat het makkelijk
% kan opgenomen worden in de bijlagen van de bachelorproef zelf.
%---------- Inleiding ---------------------------------------------------------

\section{Introductie}%
\label{sec:introductie}

\noindent Cryptocurrencies have seen a significant increase in interest over the last few years. By now almost everyone is familiar with this type of investement and a lot of people have invested themselves. This thesis will delve deeper into the possible relationship between social media and cryptocurrencies. Using data mining and analysis tools, the goal is to investigate if social media mentions and sentiment is correlated with the price of certain cryptocurrencies.  

\noindent This thesis aims to support individual and institutional crypto investors during their investment analysis. In the event that a relationship can be found between price and certain social media data, this could be of great aid to investors in order to make better decisions. 

\noindent Cryptocurrencies are new to the investing landscape which means they have some unique differences when compared to more traditional investment options. One of the big differences is the absence of financial advisors who can steer new investors into the right direction. As of yet there are very little investment institutions which offer cryptocurrency advice and guidance to their clients. In addition to this, a lot of the more established investment news sources have a troubled relationship with cryptocurrencies and don't cover them. This leads a lot of people to social media, and the internet as a whole, in order to get their investment advice. This opens up the opportunity to analyse the data which potential investors are consuming and looking for a relationship between this data and cryptocurrency prices. In this thesis we will be analysing this potential relationship and what we can conclude from it. The main research questions being :

\begin{itemize}
    \item Is there a relationship between social media data and cryptocurrency prices ?
    \item Can social media data be used in order to predict the prices of cryptocurrencies ?Is
    \item Is social media sentiment analysis a valuable addition to the investment thesis of investors.?
\end{itemize}



\noindent Once our research is over and our main questions are answered, the goal is to have a clear report which can be issued to cryptocurrency investors. This report should offer a clear conclusion and possibly aid them in future investing. 


%---------- Stand van zaken ---------------------------------------------------

\section{State-of-the-art}%
\label{sec:state-of-the-art}

\noindent The total number of cryptocurrency investors has soared during the last few years. Between 2017 and 2022 the number of people interacting with cryptocurrencies in some way has increased from 35 to 295 million \autocite{Statista2022}. During the peak of the cryptocurrency hype the total market cap of all coins reached over 3 trillion and the price of a single bitcoin was more than 69,000 dollars \autocite{CoinGecko2022}.  But how do people actually evaluate and decide which currency they want to invest in? Existing surveys suggest that a lot of people use social media in order to form their investment thesis \autocite{Graffeo2021}. A new term has even been coined which is the ``Finfluencer''. A finfluencer is a Financial influencer, someone who influences the financial decisions of their followers \autocite{CambridgeWords2021}. Experienced investors may be shocked by the fact that people use social media in order to create their investment thesis but this shouldn't come as a surprise. The average age of a cryptocurrency investor today is 38, about 10 years younger than the average stock market investor. \autocite{Iacurci2021}. Younger investors are more familiar with social media and hav used it for most parts of their life. Reports suggest that Gen Z is 5 times more likely to get investment advice from social media and 91 \% of 18-40 year-olds report getting investment-related advice from social media \autocite{Purnell2022}. In addition to this, there are currently very few alternatives to social media for cryptocurrency investment advice. Most of the established investment institutions don't offer cryptocurrency advice and some major financial news outlets avoid the topic aswell.


\noindent This leaves us with social media like Twitter, Reddit and Youtube becoming the most popular websites for cryptocurrency discussion and investment advice. There is already an existing example of social media influencing the price of a cryptocurreny. Just recently Elon Musk tweeted an image of the Dogecoin cryptocurrency and the price increased by 12\% within a single day \autocite{Saul2022}.


% Voor literatuurverwijzingen zijn er twee belangrijke commando's:
% \autocite{KEY} => (Auteur, jaartal) Gebruik dit als de naam van de auteur
%   geen onderdeel is van de zin.
% \textcite{KEY} => Auteur (jaartal)  Gebruik dit als de auteursnaam wel een
%   functie heeft in de zin (bv. ``Uit onderzoek door Doll & Hill (1954) bleek
%   ...'')



%---------- Methodologie ------------------------------------------------------
\section{Methodologie}%
\label{sec:methodologie}


\noindent In order to arrive at a suitable conclusion, social media data should be analysed over a certain period of time and subsequently compared to the evolution in price over the same period. Once the data has been collected we can run a regression analysis in order to determine a possible correlation between the variables. The dependent variable will always be the evolution in price of the cryptocurrency. There are multiple options for the independent variable, such as:

\begin{itemize}
    \item Number of mentions concerning a cryptocurrency increase in \%
    
    \item Number of mentions concerning a cryptocurrency in comparison to last month in \%
    
    \item Amount of likes posts regarding the cryptocurrency receive 
    
\end{itemize}
\noindent For this research we will be using Twitter in order to gather the required data. Over a time period of 3 months between March and May of 2023 data will be gathered and compared to the crypto prices of the same time period. Python will be used to assess the sentiment analysis of the data, mainly the Pattern and VADER libraries. Python will also  be used to gather the data from the Twitter platform.


%---------- Verwachte resultaten ----------------------------------------------
\section{Verwacht resultaat, conclusie}%
\label{sec:verwachte_resultaten}

\noindent The main result will be the conclusions we can draw from our thorough data analysis. As we will be running a regression analysis on our data we should have a final result which indicates if there is a possible relationship and which data is dependant and to what degree. If we can conclude a relationship, then our thesis could be of great benefit to potential cryptocurrency investors. The ability to gather data from social media channels is open to everyone, which means people would be able to reproduce our steps and gain an edge and improve their investing decisions. 
.



\printbibliography[heading=bibintoc]

\end{document}